\section{Scenario Testing: Use Cases} \label{app:Appendix-UseCases}
The list of use cases carried out during the scenario testing can be found here in detail.

There will be a considerable amount of use cases, which have a similar structure, therefore we will define use case templates in the attempt to minimise unnecessary repetition. The templates will be applied on multiple scenarios, varying a single step for different input partitions. Also subsequent use cases may refer back to previously defined procedures for creating data structures.

This way we can define the use cases in a bulk fashion, but during testing, each individual use cases will be carried out in full. 

If the program is not run while reading through use cases, the figures in the user manual (Appendix \ref{app:UserManual}) can be used for component overview in frames.


%%%%%%%  Commands %%%%%%%  


\newcommand{\usecase}[2]
{
\vspace{5mm}
\textsc{Use Case:} \textbf{#1}
\vspace{1mm}
\begin{enumerate}#2\end{enumerate}
}

\newcommand{\step}[2]
{
\item \textsc{Action:} #1. \newline
\textsc{Expected:} #2.
}

\newcommand{\placeholder}
{
\item \textless\textsc{placeholder}\textgreater
}

\newcommand{\placeholders}[1]
{
List of applied placeholders:
\begin{itemize}#1\end{itemize}
}


%%%%%%%  Use Case %%%%%%%  


\usecase{Program navigation with empty model}
{
\step
{Start Program \footnote{It is expected, that all use cases have this first step, and will therefore not be written explicitly for remaining use cases}}
{Program starts on home frame}

\step
{Go to LCIA methods frame}
{Empty frame\footnote{An empty frame is interpreted as a frame containing the implemented components, but with no interaction available for the user, i.e. no sensitive/enabled components.}, with only new method button available}

\step
{Go to Background systems frame}
{Empty frame, with no button available}

\step
{Go to Foreground systems frame}
{Empty frame, with no button available}

\step
{Go to Output frame}
{Empty frame, with no defined fore- or background systems}
}


%%%%%%%  Use Case %%%%%%%  


\usecase{Creation of LCIA Method}
{
\step
{Go to LCIA methods frame}
{Empty frame, with only new method button available}

\step
{Enter name of new method in entry field and press the 'New Method' button.}
{New method is created and chosen in the combobox. Method altering buttons 'Add Category' and 'Add Multiple Categories' are enabled.}

\placeholder \footnote{The placeholder will be used for applying the use case for different input partitions.}

\step
{Press any navigation button in top ribbon}
{The unsaved method disappears, and frame is empty.}}

\placeholders{
\step
{Do Nothing}
{Method is unchanged}

\step
{Press 'Add Category' button}
{Impact category is added to method with default values 'Name' and 'Unit' and displayed in the treeview}

\step
{Press 'Add Multiple Categories' button and enter column of names in data prompt. Followed by pressing 'OK'}
{Impact categories are added to method with specified names and default unit 'Unit' and displayed in the treeview.}

\step
{Press 'Add Multiple Categories' button and enter column of names and column of units in data prompt. Followed by pressing 'OK'}
{Impact categories are added to method with specified names and units and displayed in the treeview.}}


%%%%%%%  Use Case %%%%%%%  


\usecase{Non unique category names in LCIA Method}
{
\step
{Go to LCIA methods frame}
{Empty frame, with only new method button available}

\step
{Enter name of new method in entry field and press the 'New Method' button.}
{New method is created and displayed in tree view. Method altering buttons 'Add Category' and 'Add Multiple Categories' are enabled.}

\step
{Press 'Add Category' twice}
{Same behaviour as when previously adding categories. Also 'Save Method' button will be disabled}

\step
{Alter name of categories to be unique}
{'Save Method' Button enables.}

\step
{Alter name of categories back to being non-unique}
{'Save Method' Button disables.}
}


%%%%%%%  Use Case %%%%%%%  


\usecase{Saving and Selecting LCIA Methods}
{
\step
{Go to LCIA methods frame}
{Empty frame, with only new method button available}

\placeholder

\step
{Press 'Save Method'}
{Method altering buttons disables. 'Select Method' button enabled}

\step
{Press 'Select Method'}
{Method name is main window's "Selected Method' label}
}

\placeholders{
\step
{Add no categories}
{Empty method}

\step
{Add a single category}
{Method with one category}

\step
{Add multiple categories}
{Method with multiple categories}
}


%%%%%%%  Use Case %%%%%%%  


\usecase{Creating Background Systems}
{
\step
{Create and select method in LCIA method frame} \footnote{For subsequent use cases regarding system frames, it is assumed that an LCIA method is defined. This will therefore not be written expicitely.}
{Method name is main window's "Selected Method' label}

\step
{Go to Background systems frame}
{'New Background System' button is enabled}

\step
{Enter new background system name in entry and press 'New Background System'}
{Background system is created, chosen in the combobox. Buttons 'New Unit Process', 'Add Background System Data' and 'Remove Background System' are enabled}

\placeholder

\step
{Press 'Add Background System Data'}
{Data Window opens for the chosen background system. General dynamic property and unit processes usages are shown. If background system contains no unit processes respective components are disabled.}
}

\placeholders{
\step
{Do nothing}
{Empty background system}

\step
{Create new unit processes}
{Unit processes name with default usage of 1 is shown in tree view below.}

\step
{Create multiple unit processes}
{Multiple unit processes are shown in tree view below}

\step
{Create unit process, choose the unit process and press 'Remove Unit Process'}
{Unit process is created as previously, and removed again from background system}
}


%%%%%%%  Use Case %%%%%%%  


\usecase{Add General Dynamic Property to Background System}
{
\step
{Create a background system as previously defined}
{Background system is shown in combobox and has defined unit processes}

\step
{Press 'Add Background System Data'}
{Data prompt opens with no chosen unit process in combobox and model altering buttons disabled except 'Add Dynamic Property'}

\step
{Press 'Add Dynamic Property'}
{Empty function expression is added in treeview to the right in default time interval [0,0]}

\placeholder

\step
{Press 'Apply'}
{Data Window closes and changes are added to background system}

\step
{Press 'Add Background System Data'}
{The defined general dynamic property is shown in treeview}
}

\placeholders
{
\step
{Do nothing}
{General dynamic property contains empty expression in default interval}

\step
{Add function expression with wrong syntax, and press enter}
{Function expression is ignored and left empty}

\step
{Add function expression with correct syntax, and press enter}
{Function expression is added to the general dynamic property and shown in the treeview}

\step
{Change function expression time interval to end < start}
{Interval is stored, but pressing 'Apply' is ignored}

\step
{Change function expression time interval to start <= end}
{Interval is stored}
}


%%%%%%%  Use Case %%%%%%%  


\usecase{Change General Dynamic Property of Background system}
{
\step
{Create a background system as previously defined, but with defined general dynamic property}
{Background system is shown in combobox}

\step
{Press 'Add Background System Data'}
{The defined general dynamic property is shown in treeview}

\placeholder

\step
{Press 'Apply'}
{Data Window closes and changes are added to background system}

\step
{Press 'Add Background System Data'}
{The defined general dynamic property is shown in treeview}
}

\placeholders
{
\step
{Select function and press 'Remove Dynamic Property'}
{Function expression is removed}

\step
{Change interval}
{Interval is changed}

\step
{Change function expression}
{Function expression is changed if correct syntax}
}

%%%%%%%  Use Case %%%%%%%  


\usecase{Add Usages to Background System Unit Processes}
{
\step
{Create a background system as previously defined}
{Background system is shown in combobox and has defined unit processes}

\step
{Press 'Add Background System Data'}
{Data prompt opens with no chosen unit process in combobox and model altering buttons disabled except 'Add Dynamic Property'}

\step
{Choose a unit process in the combobox}
{Unit process default usage is shown in treeview. Interpolation checkbox is given the default value for the unit process 'ticked'. Buttons 'Add Usage' and 'Add Multiple Usages' are enabled.}

\placeholder

\step
{Press 'Apply'}
{Data Window closes and changes are added to background system}

\step
{Press 'Add Background System Data'}
{The usage data defined are shown in the treeview when unit process is chosen in the combobox}
}

\placeholders
{
\step
{No new data is added}
{Only default value for unit process}

\step
{Default value is chosen and removed by pressing 'Remove Usage' button}
{Default value is reapplied when usage for component is requested}

\step
{'Add Usage' is pressed}
{New usage is added to component with timestamp 0 and value 0. Usages are shown in treeview}

\step
{Usage timestamp and value are changed to wrong syntax (not decimal number)}
{New values are ignored}

\step
{Usage timestamp and value are changed to correct syntax (decimal number)}
{New values are added to usage}

\step
{'Add Multiple Usages' is pressed, and column of values are defined in data window. Press button 'OK' pressed}
{Usages are added with defined values as timestamps and value 0}

\step
{'Add Multiple Usages' is pressed, and two columns of values are defined in data window. Press button 'OK' pressed}
{Usages are added with defined values as timestamps and values respectively}

\step
{Usages are added with same timestamp}
{Subsequent usages definitions overwrite usage value at given timestamp after pressing 'Apply'}
}


%%%%%%%  Use Case %%%%%%%  


\usecase{Remove Background System}
{
\step
{Create a background system as previously defined}
{Background system is shown in combobox and has defined unit processes}

\step
{Press 'Remove Background System'}
{Background system is removed from frame. If other background systems are defined, they will be shown, otherwise an empty frame is seen}
}


%%%%%%%  Use Case %%%%%%%  


\usecase{Create Foreground System}
{
\step
{Go to Foreground systems frame}
{'New Foreground System' button is enabled}

\step
{Enter new foreground system name in entry and press 'New Foreground System'}
{Foreground system is created, chosen in the combobox. Buttons 'New Unit Process', 'Add Foreground System Data' and 'Remove Foreground System'  are enabled}

\placeholder

\step
{Press 'Add Foreground System Data'}
{
Data Window opens for the chosen foreground system. 
Shown components are: Unit processes, background systems, impact categories and general dynamic property each with their dynamic properties. If foreground system contains no unit processes or background systems, the respective components are disabled.}
}

\placeholders{
\step
{Do nothing}
{Empty foreground system}

\step
{Create unit processes similar to with background systems}
{Unit processes names with default usages of 1 are shown in tree view below.}

\step
{Create Background Systems as previously defined, and add to foreground system by pressing 'Add Background System'}
{Background systems are shown in bottom treeview, displaying name and default usages of 1}

\step
{Create and Add background systems as previous. Remove Background system by choosing it and pressing 'Remove Background System'}
{Background system as added as previousm and removed again from foreground system.}
}


%%%%%%%  Use Case %%%%%%%  


\usecase{Add/Change General Dynamic Property in Foreground System}
{
\step{Similar procedure as with background system}
{Similar behaviour}
}


%%%%%%%  Use Case %%%%%%%  


\usecase{Add/Change Usages to Foreground System's Unit Processes}
{
\step{Similar procedure as with background system}
{Similar behaviour}
}


%%%%%%%  Use Case %%%%%%%  


\usecase{Add/Change Usages to Foreground System's Background Systems}
{
\step{Similar procedure as with unit processes}
{Similar behaviour}
}


%%%%%%%  Use Case %%%%%%%  


\usecase{Add/Change Dynamic Property to Foreground System's Background Systems}
{
\step{Similar procedure as with general dynamic properties when background system is chosen in combobox}
{Similar behaviour}
}


%%%%%%%  Use Case %%%%%%%  


\usecase{Add/Change Dynamic Property to Foreground System's Impact Categories}
{
\step{Similar procedure as with general dynamic properties when category is chosen in combobox}
{Similar behaviour}
}


%%%%%%%  Use Case %%%%%%%  


\usecase{Add Dynamic Property to Unit Process}
{
\step
{Create system with the unit process as previously defined (background or foreground)}
{System is created and shown in the frame with the unit process added}

\step
{Choose unit process and press 'Add Unit Process Data'}
{Data window is opened with dynamic property for unit process (empty by default)}

\step
{Add/Change dynamic property as previously defined}
{Similar behaviour as previously defined}
}


%%%%%%%  Use Case %%%%%%%  


\usecase{Add Dynamic Property to Impact Quantity}
{
\step
{Create system with the unit process as previously defined (background or foreground)}
{System is created and shown in the frame with the unit process added}

\step
{Choose unit process}
{Impact quantities corresponding to impact categories in selected LCIA method are shown in treeview}

\step
{Choose an impact quantity and press 'Add Impact Data'}
{Data window is opened with impact quantities (timestamp 0 and value 0 by default) and dynamic property for the chosen impact quantity}

\step
{Similar procedure as with general dynamic properties}
{Similar behaviour}
}



%%%%%%%  Use Case %%%%%%%  


\usecase{Add Quantity Data Values to Impact Quantity}
{
\step
{Create system with the unit process as previously defined (background or foreground)}
{System is created and shown in the frame with the unit process added}

\step
{Choose unit process}
{Impact quantities corresponding to impact categories in selected LCIA method are shown in treeview}

\step
{Choose an impact quantity and press 'Add Impact Data'}
{Data window is opened with impact quantities (timestamp 0 and value 0 by default) and dynamic property for the chosen impact quantity}

\step
{Similar procedure as with usages for background systems/unit processes}
{Similar behaviour}
}



%%%%%%%  Use Case %%%%%%%  


\usecase{Output frame With Defined Foreground Systems}
{
\step{Create foreground systems as previously defined and navigate to Output frame}
{Output frame is shown with the type-combobox set on 'Foreground Systems' and component-combobox set on first defined foreground system in selected LCIA method, and category-combobox set on first defined impact category (if any)}

\step
{Change type-combobox to 'Background Systems'}
{Remainig frame is empty, as no background systems are defined}

\step
{Change type-combobox to 'Foreground Systems'}
{Similar behaviour as previous}

\step
{Choose impact category in category-combobox}
{Buttons 'Calculate' and 'Print Output' are enabled.}

\step
{Press 'Print Output' and choose file for saving to in file chooser window}
{Empty file is created, as no results are calculated yet}

\step
{Press 'Calculate'}
{Plot is updated with the components indicator scores according to defined impact calculation algorithm. The value for the chosen category is shown.}

\step
{Press 'Print Output' and choose file for saving to in file chooser window}
{File is created/updated with indicator scores for the chosen component for all timestamps defined from start to end timestamp separated by the interval size. Default values are start 0, interval size 1 and end 100}

\step
{Change interval start, size and end}
{Values are updated if correct syntax (decimal numbers)}

\step
{Calculate and Print Output}
{Values plotted and saved are updated with specified interval properties}
}



%%%%%%%  Use Case %%%%%%%  


\usecase{Output frame With Defined Background Systems}
{
\step
{Similar to procedure with foreground systems}
{Similar behaviour}
}



%%%%%%%  Use Case %%%%%%%  


\usecase{Save/Load}
{
\step
{Create a model with procedures defined previous. Go to Home frame and Save}
{Model is saved in chosen file}

\step
{Close program and start again}
{Program closes and opens}

\step
{Load file saved previously}
{Content is loaded, and seen on method, background system and foreground system frames}
}
