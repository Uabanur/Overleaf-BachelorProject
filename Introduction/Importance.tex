\section{Importance of Dynamic Modelling}\label{sec:Importance}

As an extension of the existing LCA tools, we will look at the influence of dynamic modelling when assessing production systems. It seems intuitive, that an active system during a time period, is assessed more accurately when introducing dynamic modelling, and we will look into some studies which incorporate these concepts, and the consequences thereof. 

An important thing to keep in mind, is that the results from the impact assessment (apart from showing if the intended system is environmentally appropriate), guide the decision maker between different choices for the production system. 

Such choices are mainly based on which setup yields the least impact, and therefore the most important feature of such a tool, is to make it possible to compare different setups. This is especially true, when we model dynamic systems, as the impact may change over time, and the accumulated impact for the systems lifetime is the deciding factor. 

With dynamic modelling, the impact assessment will also provide information for the decision maker regarding the lifetime for the system, both regarding growth and if it is more beneficial to repair or substitute components. Several financial decisions may be clarified, making it not only a natural extension of the existing software, but also a potentially strong tool for any LCA model interpretation. 

\subsection{Effects of Dynamic Modelling} \label{sec:EffectsDLCA}

A study of insulation material in danish residential buildings was conducted, analysing the relationship between the amount of insulation used in the construction of the buildings in relation to the the heating required to keep the house warm during its lifetime. The results of the analysis indicate that the otherwise recommended level of insulation levels for the building construction was non-optimal when evaluating the political projections of the energy supply throughout the service life for the buildings. 

The optimal insulation level found yielded approximately 6147 tons less carbon dioxide equivalent annually, across the estimated amount of constructed houses per year, compared to the suggested insulation levels at the time.\cite{DREM_Analysis}

Another (in press) study analyses the current literature on dynamic LCA, and how the field is evolving currently. Results show, that the attention in this field is 'rapidly increasing', and particularly the implementation of partially dynamic LCA, implementing dynamic concepts on isolated parts of the procedure. 
It is stressed, that caution is needed, to avoid introducing unwanted biases, resulting in misleading assessments. It is recommended that the dynamic extrapolation is used for time intervals of no known data, and not to replace data.\cite{DLCALitteratureAnalysis}