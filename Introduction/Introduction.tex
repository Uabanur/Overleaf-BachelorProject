\chapter{Introduction}\label{chap:Introduction}

The intention of this report, is to introduce some extensions to the Life Cycle Assessment procedure, to more accurately evaluate a production system and its environmental impacts during its service period. First we will introduce what LCA in its simplest form is, using an example to introduce common terminology for the domain. Terminology specific for LCA will be \emph{emphasised} and a definition of the term can be found in Section \ref{sec:Terminology}.

Several tools exist at present to perform an LCA study. We will look at some of the most well-known, as well as where they fall short. 

At this point in the report it should be clear what is currently possible, using the available tools for LCA. We will then describe our overall intention with this project and its importance within the field of LCA studies.

Once the specifics of the program are defined, we will go through an initial analysis of some design aspects. There are countless possibilities when it comes to program design, and we will have to face some fundamental decisions before we start designing the program architecture. After the initial analysis we will look at the general layout of how the program will be structured.

During the implementation, several decisions have to be made, both trivial decisions (such as naming conventions) and other more crucial (such as tools). The more relevant decisions will be discussed, and evaluated. 

In order to evaluate the implementation, several forms of tests are executed. This program deals with the domain of environmental impact assessment, and it is therefore essential that it behaves as expected. 

Lastly there will be a conclusion and evaluation of the completed project.

In the appendix it is possible to see the user manual for the implemented program, as well as use cases for scenario testing and class diagrams for all implemented classes.